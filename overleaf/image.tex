\documentclass{article}

\usepackage{graphicx}
\usepackage{blindtext}
\usepackage{wrapfig2}



\author{Logosevens}
\title{The Logic of Images and Figure in {\LaTeX}}

\begin{document}

\maketitle

Here is a new document.

\begin{center}
\includegraphics[width=3in,height=1in, keepaspectratio]{fcpx.png}
\includegraphics[width=0.25\textwidth, angle=45]{fcpx.png}
\end{center}

\blindtext
\blindtext
\blindtext

In the figure below, you can see the medical effects of emacs.

\blindtext

% [t, b, h, p]
\begin{figure}[h]
\centering

\includegraphics[width=0.7\textwidth]{gojo.jpeg}

\caption{Gojo Satoru}
\end{figure}

\blindtext
\begin{wrapfigure}{r}{0.5\textwidth}
\includegraphics[width=0.5\textwidth]{gojo.jpeg}
\caption{another Gojo?\label{latexpic}}
\end{wrapfigure}

Please refer to Figure \ref{latexpic}. (This number is auto generated.)
\blindtext
\blindtext

In physics, the mass-energy equivalence is stated by the equation $E=mc^2$, discovered in 1905 by Albert Einstein.

\begin{math}
E=mc^2
\end{math} is typeset in a paragraph using inline math mode---as is $E=mc^2$, and so too is \(E=mc^2\).

The mass-energy equivalence is described by the famous equation \[ E=mc^2 \] discovered in 1905 by Albert Einstein.

In natural units ($c = 1$), the formula expresses the identity
\begin{equation}
E=m
\end{equation}

Subscripts in math mode are written as $a_b$ and superscripts are written as $a^b$. These can be combinded and nested to write expressions such as

\[ T^{i_1 i_2 \dots i_p}_{j_1 j_2 \dots j_q} =
 T(x^{i_1},\dots,x^{i_p},e_{j_1},\dots,e_{j_q}) \]

 We write integrals using $\int$ and fractions using $\frac{a}{b}$, Limits are placed on integrals using superscripts and subscripts:

 \[ \int_0^1 \frac{dx}{e^x} = \frac{e-1}{e} \]

 Lower case Greek letters are written as $\omega$ $\delta$ etc.
 while upper case Greek letters are written as $\Omega$ $\Delta$.

 Mathmatical operators are prefixed with a backslash as $\sin(\beta)$, $\cos(\alpha)$, $\log(x)$ etc.

 The well-known Pythagorean theorem \(x^2 + y^2 = z^2\) was proved to be invalid for other exponents, meaning the next equation has no integer solutions for \(n>2\):

 \[ x^n + y^n = z^n \]

 \section{Second example}

 This is a simple math expression \(\sqrt{x^2+1}\) inside text.
 And this is also the same:
 \begin{math}
   \sqrt{x^2+1}
 \end{math}
 but by using another command.

   
\end{document}
